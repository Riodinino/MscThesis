%Last page
\newpage
\scriptsize{
\paragraph{Résumé :}
Les forêts tropicales abritent la moitié de la biodiversité terrestre mondiale, fournissent d'importants services écosystémiques à l'humanité, et sont un important réservoir de carbone. Ces forêts font face à de nombreuses menaces, dont la déforestation à des fins agricoles, et l'exploitation forestière séléctive. Cette dernière a affecté ou affectera la majorité des forêts tropicales, et a longtemps été une pratique incontrôlée. La Gestion Forestière Durable a été mise en avant pour tenter de résoudre ce problème, s'appuyant notamment sur les principes de l'Exploitation à Faible Impact et des incitations financières, comme le programme REDD+. Cependant, certains auteurs ont remis en question la durabilité réelle d'une telle exploitation. Evaluer l'impact des pratiques forestières est une tâche difficile, au vu des échelles de temps impliquées. En complément des efforts de suivi, la modélisation peut s'avérer utile pour fournir un aperçu des effets de l'exploitation forestière à plus long terme.   
TROLL est un modèle spatialement explicite, individu-centré, qui simule un grand nombre d'espèces d'arbres. TROLL offre d'intéressantes perspectives en écologie théorique et appliquée. Nous avons exploré le potentiel de ce modèle pour simuler l'exploitation sélective, et explorer la durabilité de plusieurs scénarios.  Nous avons commencé par paramétrer 547 espèces à partir de la base de données BRIDGE, pour simuler d'importantes étendues de forêt, et avons ensuite essayé de réaliser une évaluation et calibration des trajectoires post-coupe simulées, en partaint de données réelles. La calibration fut impossible à cause d'une mortalité éxagérée dans les forêts d'entrées, pendant les premières années de simulation. Nos analyses suggèrent que TROLL a une structure spatiale différente de celle de vraies forêts, peut être à cause d'une sur-estimation de la compétition pour la lumière, le rendant inadapté à simuler à partir de vraies données pour le moment. Nous avons adapté la version existante du module simulant l'exploitation dans TROLL, pour implémenter plusieurs pratiques et paramètres sylvicoles. Nous avons réalisé un premier jeu de simulations sur deux forêts ayant des volumes de départ contrastés, durant 5 rotations. Nos résultats indiquent que l'exploitation sélective, telle qu'elle est pratiquée en Guyane Française, pourrait mener à un épuisement des volumes totaux d'espèces commerciales, ainsi qu'a une baisse du stock de carbone au fil des récoltes, et cela même pour des rotations de 65 ans, avec des intensités de 20$m^3$ et les techniques de l'EFI. Ces résultats sont cependant préliminaires et manquent de réplication, ils doivent donc être interprétés avec précaution. Des analyses plus poussées sur des scénarios plus nombreux sont requises afin de confirmer et affiner nos résultats. TROLL, combiné avec le modèle d'exploitation que nous avons mis à jour, présente un potentiel promettent pour explorer différents scénarios sylvicoles, et répondre à des problématiques appliquées.
\paragraph{Mots clés :} Gestion Forestière Durable, Simulation, Exploitation forestière, Modélisation, Modèle individu-centré.
\paragraph{Abstract:}
Tropical forests shelter half the terrestrial biodiversity worldwide, provide important services to humanity and are a major reservoir of carbon. These forests face numerous threats, among others deforestation for agriculture and selective logging. Selective logging affected or will affect the majority of tropical forest outside protected areas, and has long been an uncontroled predatory practice. Sustainable Forest Management (SFM) has been promoted to answer this issue, relying on Reduced Impact Logging, and financial incentives such ad REDD+. However, some authors asked whether these techniques are sustainable. Assessing the sustainability of Forest Management difficult task, because of the efforts needed by field studies, and time scales involved. To complement monitoring efforts, the use of models can provide valuable insights of longer-term effects of selective logging. TROLL is an individual-tree-based, spatially explicit forest model that uses functional traits to simulate the life cycle of a wide range of tree species. TROLL offers promising perspectives in studying ecological theories and applied problems. We explored the potential of this model to simulate selective logging and assess the sustainability of different scenarios. We preliminarily parametrized 547 species from the BRIDGE dataset to simulate large forest plots, and we attempted to evaluate and calibrate the simulated post-logging trajectories inputting real forest censuses in the model. The calibration was impossible because of exagerated mortality in the inputted forests during the first years of simulation. Spatial structure analysis suggest that TROLL has a different spatial structure than real forests, maybe due to overestimation of competition for light, thus making unadapted the use of real data inputs for now. We adapted the existing version of the module that simulates selective logging in TROLL, to implement cutting cycles, conventional logging, and designation based on timber interest ranks. We did a preliminary set of simulations on two forests that had contrasted timber volumes, to assess the importance of silviculture parameters over five cutting cycles. Our results indicate that selective logging, as applied in French Guiana and neighboring countries, may lead to a depletion of total timber stocks and high-grade species, along with a decrease of carbon stocks over harvests, even for 65 years cutting cycles with 20$m^3$ harvested and RIL techniques. However, our results are preliminary, lack replication, and thus have to be interpreted carefully. Further, extensive analyses are needed to confirm and refine our findings. TROLL, combined with the logging model we updated, has a promising potential to explore a wide range of silviculture scenarios, and adress applied problematics.
\paragraph{Keywords:}Sustainable Forest Management, Simulation, Selective logging, Modeling, Individual-based model.
}

\vspace*{\fill}
\includegraphics{images/logo}